\documentclass[12pt,a4j,dvipdfmx]{jarticle}
\usepackage{url}  % \url 命令を利用
\usepackage{otf}
\title{p\LaTeXe をもっと便利に}
\author{朝倉卓人\thanks{e-mail: \texttt{wtsnjp@gmail.com}}}

\begin{document}
\maketitle

\section*{\LaTeX 関連の情報や成果物を入手しよう}
近年人気のディストリビューション(\TeX\ Live等)を利用している場合,
自力で入手しなくとも数多くのクラスやパッケージが最初からインストール
されている.

より多くの情報や成果物を入手するには表\ref{表:サイト集}掲載のサイトを
訪れるとよい.

\begin{table}[bth!]
\centering
\caption{\emph{有益なサイト}.英語のサイトも含まれている.}
\label{表:サイト集}
\begin{tabular}{lll}
\emph{サイト名} & \emph{URL} & \emph{説明} \\ \hline
\TeX\ Wiki & \url{https://texwiki.texjp.org/} & 日本語情報の宝庫 \\
TUG & \url{https://www.tug.org/} & \TeX\ Users Groupのサイト \\
CTAN & \url{https://www.ctan.org/} & 世界から成果物が集まる \\
\end{tabular}
\end{table}
\end{document}
